\documentclass[oneside,final,14pt]{extreport}
\usepackage[utf8]{inputenc}
\usepackage[russianb]{babel}
\usepackage{vmargin}
\setpapersize{A4}
\setmarginsrb{2cm}{1.5cm}{1cm}{1.5cm}{0pt}{0mm}{0pt}{13mm}
\sloppy
\usepackage{indentfirst}
\usepackage{graphicx}
\usepackage{amsmath}
\begin{document}

\begin{titlepage}
\begin{figure}[t]
	\centering
	\includegraphics[width=0.75\textwidth]{top}
\end{figure}
\begin{centering}
Московский государственный университет имени М.В. Ломоносова\\
Факультет Вычислительной математики и кибернетики\\
Кафедра Суперкомпьютеров и квантовой информатики\\
\vfill
\vfill
\vfill
\LARGE
{\bf ОТЧЁТ}
\vfill
\Large
Параллельная реализация преобразования Хаусхолдера для решения СЛАУ на IBM Blue Gene

\end{centering}
\normalsize
\vfill
\vfill
\vfill
\vfill
\begin{flushright}
Выполнил:\\
студент 423 группы\\
Ромащенко Константин Олегович
\vfill
\end{flushright}
\vfill
\vfill
\vfill
\vfill
\vfill
\centerline{Москва, 2017}

\end{titlepage}

\section*{Постановка задачи}

Задача заключается в том, чтобы:
\begin{enumerate}
\item Реализовать алгоритм решения системы линейных алгебраических уравнений с помощью преобразования Хаусхолдера в виде последовательной и параллельных
программ
\item Запустить программы на IBM Blue Gene, замерить время выполнения при различном количестве задействованных процессоров для параллельной программы
\item Составить таблицы и графики, из которых сделать выводы о распараллеливаемости программы
\end{enumerate}

\section*{Описание алгоритма}

\subsection*{Математический базис}

Алгоритм решает СЛАУ
\begin{equation}
Ax=b,
\end{equation}
где $A$ ~--- это матрица системы, $x$ ~--- столбец неизвестных, $b$ ~--- столбец свободных членов.

Сначала матрица $A$ приводится к вернему треугольному виду. Для этого строятся матрицы
\begin{equation}
U(x) = I - 2 x x^*,
\end{equation}
где $I$ ~--- единичная матрица, $x^*$ ~--- вектор, сопряжённый вектору $x$.

Эта матрица обладает следующими свойствами:
\begin{enumerate}

\item Если вектор $x$ нормирован, то
\begin{equation}\label{eq1}
U(x) x = I x - 2 x x^* x = x - 2 x = -x
\end{equation}

\item $ \forall y \in x^\perp $
\begin{equation}\label{eq2}
U(x) y = I y - 2 x x^* y = y
\end{equation}

\end{enumerate}

Пусть $a_i$ ~--- $i$-й столбец матрицы $A$. Тогда для каждого столбца матрицы $A$, кроме последнего, берётся вектор $a_i=(a_{ii}, a_{i(i+1)} ... a_{in})^T$ 
и строится матрица
\begin{equation}
U_i = \begin{bmatrix}
I_{(i-1)} & O\\
O & U(x^{(i)})
\end{bmatrix}
\end{equation}
где
\begin{equation}\label{eq3}
x^{(i)} = \frac{a_i - \| a_i \| e_i}{\| a_i - \| a_i \| e_i \|},
\end{equation}

$e_i$ ~--- $i$-й базисный вектор, $O$ ~--- нулевая матрица, $I$ ~--- единичная матрица размера $(i-1)$ (если $i-1=0$, то этой и нулевой матриц просто нет, тогда 
$U_1 = U(x^{(1)})$)

После этого СЛАУ домножается слева на $U_i$
\begin{equation}
A^{(i)} = U_i A^{(i-1)},
b^{(i)} = U_i A^{(i-1)},
A^{(i)} x = b^{(i)},
\end{equation}
при этом положим $A^{(0)} = A$, $b^{(0)} = b$.

С учётом свойств \ref{eq1} и \ref{eq2} получаем, что при умножении $U_i$ на $A^{(i-1)}$ обнуляется $i$-й столбец ниже $i$-го элемента.
В результате получается верхняя треугольная матрица.

Затем производится обратный ход как в методе Гаусса, что завершает решение СЛАУ.

\subsection*{Особенности параллельной реализации}

Ввиду того, что умножение матриц независимо по столбцам, матрицу $A$ можно разбить на столбцы и выполнять умножения одновременно. Таким образом, вся матрица
разбивается на столбцы по памяти процессам, также считывается в один из процессов вектор $b$. На каждой итерации процесс, содержащий $i$-й столбец, 
считает по формуле \ref{eq3} вектор $x^{(i)}$ и рассылаем вектор всем остальным процессам. Получившие данный вектор процессы уже выполняют умножение на столбцы.

\section*{Компиляция и запуск программы}

\section*{Результаты запуска}

\section*{Заключение}

\end{document}
